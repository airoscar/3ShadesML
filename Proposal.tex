\documentclass[conference]{IEEEtran}
\IEEEoverridecommandlockouts
% The preceding line is only needed to identify funding in the first footnote. If that is unneeded, please comment it out.
\usepackage{cite}
\usepackage{amsmath,amssymb,amsfonts}
\usepackage{algorithmic}
\usepackage{graphicx}
\usepackage{textcomp}
\usepackage{xcolor}
\def\BibTeX{{\rm B\kern-.05em{\sc i\kern-.025em b}\kern-.08em
    T\kern-.1667em\lower.7ex\hbox{E}\kern-.125emX}}
\begin{document}

\title{Proposal for Fast Facial Detection and Recognition \\
}

\author{\IEEEauthorblockN{Oscar Chen}
\IEEEauthorblockA{\textit{Dept of Elec \& Comp Engineering} \\
\textit{University of Calgary}\\
Calgary, Canada \\
oscar.chen1@ucalgary.ca}
\and
\IEEEauthorblockN{Savith Jayasekera}
\IEEEauthorblockA{\textit{Dept of Elec \& Comp Engineering} \\
\textit{University of Calgary}\\
Calgary, Canada \\
scjayase@ucalgary.ca}
\and
\IEEEauthorblockN{Prince Okoli}
\IEEEauthorblockA{\textit{Dept of Elec \& Comp Engineering} \\
\textit{University of Calgary}\\
Calgary, Canada \\
prince.okoli@ucalgary.ca}

}

\maketitle

\begin{abstract}

We propose to combine a Viola-Jones feature-based detector with techniques that use Convolutional Neural Network, to construct a method to detect, recognize and differentiate people from streaming video footage. We want to leverage Viola-Jones feature based detector for fast detection of capturing human faces from video footage, and further process the cropped facial images by treating it as a regression problem.

\end{abstract}

\begin{IEEEkeywords}
component, formatting, style, styling, insert
\end{IEEEkeywords}

\section{Introduction}

Facial detection and recognition is a relative old area in computer vision. Convolutional neural network (CNN) has been used in the past to achieve good object detection capabilities. With large enough sample set, CNN can perform classifying and generalizing tasks with high accuracy as compared to many other techniques. However CNN is traditionally slow and unsuited for performing classifying tasks on live streaming video footage. CNN also requires large amount of image to train for each class, therefore making it unfeasible as a classifier for facial recognition. By treating facial recognition as a regression, one can assign a face to a vector and thus not requiring training on large set of images for each face. When Viola-Jones feature detector used in conjunction with such a CNN, we believe fast facial recognition and differentiation can be achieved without prior training on each subject.

\section{Related Work}

\section{Methodology}

\subsection{Research Questions}
\subsection{Data Collection and Preparation Plan}
\subsection{Data Analysis Plan}


\section{Expected Results}
\section{Discussion}
\subsection{Limitations}
\subsection{Possible Extensions [Future Work]}
\section{Conclusion}


\end{document}
